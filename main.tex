\documentclass{article}
\usepackage{graphicx} % Required for inserting images
\usepackage{enumitem}
\usepackage{amsmath} % Para símbolos matemáticos
\usepackage{array} % Para melhorar a formatação da tabela

\title{ATIVIDADE REALIZADA ATRAVÉS DO LATEX}
\author{Ryan Paulino}
\date{19/03/2025}

\begin{document}

\maketitle


\section{Lógica Proposicional e Aplicações}

\textbf{1. (Rosen 1.1.1) Quais das sentenças abaixo são proposições? Qual o valor de verdade das que são proposições?}
\begin{itemize}[label={}]
    \item (a) Curitiba é a capital do Paraná.
    \item (b) Joinville é a capital de Santa Catarina.
    \item (c) 2 + 3 = 5
    \item (d) 5 + 7 = 10
    \item (e) x + 2 = 11 \> Não é uma proposição, pois o x é desconhecido, sendo assim, pode ser verdade ou não.
    \item (f) Responda a esta questão. \> Não é uma proposição, pois é uma ordenação.
\end{itemize}

\begin{table}[h]
    \centering
    \begin{tabular}{c|c}
    \hline
    Sentenças & Valor de Verdade \\
       A   & Verdadeira \\
       B   & Falsa \\
       C   & Verdadeira \\
       D  & Falsa \\
    \hline
       
    \end{tabular}
\end{table}

    \vspace{1cm}
    \textbf{2. (Rosen 1.1.5) Considere que p e q são as proposições: “Nadar na praia em Copacabana é permitido” e “Foram descobertos tubarões perto da praia”, respectivamente. Expresse cada uma dessas proposições compostas como uma sentença em português.}

    \begin{itemize}[label={}]
       \item(a) $\neg  q$ \quad \= "Nadar na praia em copacabana não é permitido" "Não foram descobertos tubarões perto da praia" \\
       
       \item(b) $p \wedge q$ \> "Nadar na praia em Copacabana é permitido e Foram descobertos tubarões perto da praia" \\
       
       \item(c) $\neg p \vee q$ \> "Nadar na praia em copacabana não é permitido ou Foram descobertos tubarões perto da praia" \\
       
       \item(d) $p \rightarrow \neg q$ \> "Se nadar na praia em Copacabana é permitido, então não foram descobertos tubarões perto da praia" \\
       
       \item(e) $p \leftrightarrow \neg q$ \> "Nadar na praia em Copacabana é permitido somente se não for descoberto tubarões perto da praia" \\
       
       \item(f) $\neg p \wedge (p \vee \neg q)$ \> "Nadar na praia em Copacabana não é permitido e, ou Nadar na praia em Copacabana é permitido, ou não foram descobertos tubarões perto da praia." \\
    \end{itemize}
    
\vspace{1cm}

\textbf{3. (Rosen 1.1.7) Considere que p e q são as proposições:}

   \begin{itemize}[label={}]
    \item \( p \): "Está abaixo de zero."
    \item \( q \): "Está nevando."
\end{itemize}

\textbf{Escreva estas proposições usando p, q e conectivos lógicos.}

\begin{itemize}[label={}]
    \item(a) Está abaixo de zero e nevando: \> $p \wedge q$ \\
        
    \item(b) Está abaixo de zero, mas não está nevando: \> $p \wedge \neg q$
    
    \item(c) Não está abaixo de zero e não está nevando: \> $\neg p \wedge \neg q$
    
    \item(d) Está nevando ou abaixo de zero (ou os dois): \> $q \lor p$
    
    \item(e) Se está abaixo de zero, está também nevando: \> $p \rightarrow q$
    
    \item(f) Está ou nevando ou abaixo de zero, mas não está nevando se estiver abaixo de zero: \> ($p \vee  q) \wedge (p \rightarrow \neg q)$
    
    \item(g) Para que esteja nevando, é necessário e suficiente que esteja abaixo de zero: \> $q \leftrightarrow p$
    
\end{itemize}

\vspace{1cm}

\textbf{4. (Rosen 1.1.11) Sejam p, q e r as seguintes proposições:}

    \begin{itemize}[label={}]
        \item \textbf{p}: “Ursos-cinzentos são vistos na área.”
        \item \textbf{q}: “Fazer caminhada na trilha é seguro.”
        \item \textbf{r}: “As bagas estão maduras ao longo da trilha.” 
    \end{itemize}

    \begin{itemize}[label={}]
    \vspace{0.5mm}
        \item \textbf{Escreva as seguintes proposições utilizando p, q, r e conectivos lógicos.}
    \end{itemize}


\begin{itemize}[label={}]
    \item (a) As bagas estão maduras ao longo da trilha, mas os ursos-cinzentos não são vistos na área: \> $r \wedge \neg p$
    
    \item (b) Ursos-cinzentos não são vistos na área e fazer caminhada na trilha é seguro, mas as bagas estão maduras ao longo da trilha: \> $\neg p \wedge q \wedge r$
    
    \item (c) Se as bagas estão maduras ao longo da trilha, fazer caminhada é seguro se, e somente se, ursos-cinzentos não forem vistos na área: \> $r \rightarrow (q \leftrightarrow \neg p)$
    
    \item (d) Não é seguro fazer caminhada na trilha, mas os ursos-cinzentos não são vistos na área e as bagas ao longo da trilha estão maduras: \> $\neg q \wedge \neg p \wedge r $
    
    \item (e) Para a caminhada ser segura, é necessário, mas não suficiente, que as bagas não estejam maduras ao longo da trilha e que os ursos-cinzentos não sejam vistos na área: \> $q \rightarrow (\neg r \wedge \neg p)$

    \item (f) Caminhada não é segura ao longo da trilha sempre que os ursos-cinzentos são vistos na área e as bagas estão maduras ao longo da trilha: \> $(p \wedge r)  \rightarrow \neg q $
    
\end{itemize}

\vspace{1cm}

\textbf{5. (Rosen 1.1.13) Determine se cada uma destas proposições condicionais é verdadeira ou falsa.}

\begin{itemize}[label={}]
    \item (a) Se 1 + 1 \texttt{=} 2, então 2 + 2 \texttt{=} 5. \> Falsa \\
    \item (b) Se 1 + 1 \texttt{=} 3, então 2 + 2 \texttt{=} 4. \> Verdadeira \\
    \item (c) Se 1 + 1 \texttt{=} 3, então 2 + 2 \texttt{=} 5. \> Verdadeira \\
    \item (d) Se macacos puderem voar, então 1 + 1 \texttt{=} 3. \> Verdadeira \\
\end{itemize}

\vspace{1cm}

\textbf{6. (Rosen 1.1.17) Reescreva cada sentença em português explicitando o que ela significa caso o “ou” utilizado seja inclusivo (ou seja, uma disjunção) e o que ela significa caso o “ou” utilizado seja exclusivo. Quais dos significados do “ou” você acha que o autor queria usar?}

\begin{itemize}[label={}]
    \item (a) Para cursar matemática discreta, você deve ter tido cálculo ou um curso de ciência da computação. \> 
    
    \textbf{R}: Seria \textbf{OU inclusivo}, pois para cursar matemática discreta, você deve ter um curso de cálculo ou de ciência da computação, ou ambos os cursos. \\
    
    \item (b) Quando você compra um novo carro da Companhia Acme Motor, você pega de volta \$2.000 ou um empréstimo de 2\%. \> 
    
    \textbf{R}: Seria o \textbf{OU exclusivo}, pois você tem que escolher um ou outro, mas não os dois. \\
\end{itemize}

\vspace{1cm}

\textbf{7. (Rosen 1.1.19) Escreva cada uma das proposições abaixo na forma “se p, então q” em português.}

\begin{itemize}[label={}]
    \item (a) Neva sempre que o vento sopra no Nordeste. \>
    
    \textbf{R}: Se o vento sopra no Nordeste, então sempre neva.\\

    
    \item (b) É necessário andar 8 quilômetros para chegar ao topo do monte Roraima. \>
    
    \textbf{R}: Se não andar 8 quilômetros, então não é possível chegar ao topo do monte Roraima.\\
    

    \item (c) Para conseguir ser efetivado como professor, é suficiente ser famoso na universidade.\>
     
    \textbf{R}: Se for famoso na universidade, então conseguirá ser efetivado como professor.\\
   
    
    \item (d) Sua garantia é válida apenas se você comprou seu aparelho de som em menos de 90 dias. \>
     
     
    \textbf{R}: Se você não comprou seu aparelho de som em menos de 90 dias, então sua garantia não é válida.\\
   
    
    \item (e) Yogananda nadará a menos que a água esteja muito fria. \>
     
    \textbf{R}: Se a água estiver muito fria, então Yogananda não nadará. \\
    
\end{itemize}

\textbf{8. (Rosen 1.1.23) Determine a oposta/conversa, a contrapositiva e a inversa de cada uma das proposições condicionais.}

\begin{itemize}[label={}]
    
    \item (a) Se nevar hoje, então esquiarei amanhã. \\ 
    \textbf{R}: 
    \item \textbf{(Conversa)}: Esquiarei amanhã, se hoje nevar.
    \item \textbf{(Contrapositiva)}: Não esquiarei amanhã, se não nevar hoje.
    \item \textbf{(Inversa)}: Se não nevar hoje, não esquiarei amanhã.
\\

    \item (b) Se há uma prova, então eu venho à aula. \\
    \textbf{R}: 
    \item \textbf{(Conversa)}: Se eu venho a aula, então hà uma prova
    \item \textbf{(Contrapositiva)}: Não vou a aula, se não tiver prova.
    \item \textbf{(Inversa)}: Se não tiver prova, então não vou à aula.
\\
\end{itemize}

\vspace{1cm}

\textbf{9. (Rosen 1.1.27) Construa a tabela da verdade para cada uma das proposições compostas abaixo.}


\section*{Tabelas Verdade}

\subsection*{(a) \( (p \lor \neg q) \rightarrow q \)}

\begin{center}
    \begin{tabular}{|c|c|c|c|c|}
        \hline
        \( p \) & \( q \) & \( \neg q \) & \( p \lor \neg q \) & \( (p \lor \neg q) \rightarrow q \) \\  
        \hline
        V & V & F & V & V \\  
        V & F & V & V & V \\  
        F & V & F & F & V \\  
        F & F & V & V & F \\  
        \hline
    \end{tabular}
\end{center}

\subsection*{(b) \( (p \rightarrow q) \leftrightarrow (\neg q \rightarrow \neg p) \)}

\begin{center}
    \begin{tabular}{|c|c|c|c|c|c|}
        \hline
        \( p \) & \( q \) & \( \neg p \) & \( \neg q \) & \( p \rightarrow q \) & \( \neg q \rightarrow \neg p \) \\  
        \hline
        V & V & F & F & V & V \\  
        V & F & F & V & F & F \\  
        F & V & V & F & V & V \\  
        F & F & V & V & V & V \\  
        \hline
    \end{tabular}
\end{center}

\subsection*{(c) \( (p \oplus q) \rightarrow (p \land q) \)}

\begin{center}
    \begin{tabular}{|c|c|c|c|c|}
        \hline
        \( p \) & \( q \) & \( p \oplus q \) & \( p \land q \) & \( (p \oplus q) \rightarrow (p \land q) \) \\  
        \hline
        V & V & F & V & V \\  
        V & F & V & F & F \\  
        F & V & V & F & F \\  
        F & F & F & F & V \\  
        \hline
    \end{tabular}
\end{center}

\end{document}

        

\end{document}
